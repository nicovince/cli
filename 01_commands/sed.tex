\subsection{sed}

\begin{frame}[fragile]{sed}
  \begin{itemize}
    \item Used to perform automatic edition on an input stream or an existing file
    \pause
    \item Stands for Stream EDitor
  \end{itemize}
\end{frame}

\begin{frame}[fragile]{sed}
  \begin{exampleblock}{Search and Replace}
    \begin{lstlisting}[showstringspaces=false,basicstyle=\tiny]
sed 's/search/replace/g'
    \end{lstlisting}
  \end{exampleblock}
  \pause

  \begin{exampleblock}{Delete line matching pattern}
    \begin{lstlisting}[showstringspaces=false,basicstyle=\tiny]
sed '/pattern on line to delete/d'
    \end{lstlisting}
  \end{exampleblock}
  \pause

  \begin{exampleblock}{Delete line or range of lines}
    \begin{lstlisting}[showstringspaces=false,basicstyle=\tiny]
sed '3d'
sed '3,5d'
    \end{lstlisting}
  \end{exampleblock}
\end{frame}

\begin{frame}[fragile]{sed}
If you want to modify an existing files, instead of having sed doing modifications on standard input you can use \emph{-i} flag
\pause
  \begin{exampleblock}{Modify file in-place}
    \begin{lstlisting}[showstringspaces=false,basicstyle=\tiny]
sed -i 's/search/replace/g' filename
sed -i.bak 's/search/replace/g' filename
    \end{lstlisting}
  \end{exampleblock}
\end{frame}

\begin{frame}[fragile]{sed}
Sed can memorize patterns on the search side to reuse them on the replace side by using escaped parenthesis :
\pause
  \begin{exampleblock}{Memorize patterns}
    \begin{lstlisting}[showstringspaces=false,basicstyle=\tiny]
echo "# hello  | sed 's/\(#*\)\(.*\)\$/\1\2 \1/'
--> # hello #
    \end{lstlisting}
  \end{exampleblock}
\end{frame}

\begin{frame}[fragile]{sed}
If you have multiple operations to perform you can either pipe them, or put them in a sed script (script.sed, with executable rights)
\pause
  \begin{exampleblock}{Sed script}
    \begin{lstlisting}[showstringspaces=false,basicstyle=\tiny]
#!/bin/sed -f
s/#/=/g
s/^\(=*\)\(.*\)/\1\2 \1/
    \end{lstlisting}
  \end{exampleblock}

\pause
And execute the script on a file
  \begin{exampleblock}{Execute Script}
    \begin{lstlisting}[showstringspaces=false,basicstyle=\tiny]
script.sed README.md
    \end{lstlisting}
  \end{exampleblock}
\end{frame}


\begin{frame}[fragile]{sed}
  \begin{itemize}
    \item\emph{/} delimiter for commands can be replaced with \emph{:,\#\%} (non exhaustive)
    \pause
    
    \item tutorial : http://www.grymoire.com/Unix/Sed.html
  \end{itemize}
\end{frame}
